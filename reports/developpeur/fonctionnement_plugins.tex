%%%%%%%%%%%%%%%%%%%%%%%%%%%%%%%%%%%%%%%%%%%%%%%%%%%%%%%%%%%%%%%%%%%%
%%%%% DOC DEVELOPPEUR : FONCTIONNEMENT DES PLUGINS %%%%%
%%%%%%%%%%%%%%%%%%%%%%%%%%%%%%%%%%%%%%%%%%%%%%%%%%%%%%%%%%%%%%%%%%%%


L'\textsc{\'Editeur d'Armure} est un logiciel composé d'un gestionnaire de plugins et de plusieurs plugins. \\


\section{GestionnairePlugins et ApplicationProjet}

Les projets GestionnairePlugins et ApplicationProjet sont la base de logiciel. Le gestionnaire de plugin est une application contenant la fonction \texttt{main.java} et le pluginManager. Il permet donc de lancer l'application, et aussi de charger des plugins. \\

ApplicationProjet, quant à lui, est le plugin initial du logiciel. Il contient les classes de l'armure, la classe de l'ArmorEditor, et les interfaces utiles pour toute l'application (afficheur, modificateur...).


\section{Plugins d'affichage}

Les armures créées dans l'\textsc{\'Editeur d'Armure} peuvent être affichées de deux façons différentes. Soit leur description apparait dans un fichier texte, soit dans une fenêtre graphique.\\

Le plugin \textsc{AffichageConsole} utilise simplement la methode \texttt{toString} sur l'instance de l'armure, et affiche les informations recueillies dans la console d'Eclipse. Le plugin \textsc{AffichageGraphique}, plus complexe, annalyse les données d'une armure pour les afficher proprement dans une fenêtre.


\section{Plugins de création}

La création d'une armure peut également se faire de deux façons différentes. Le plugin \textsc{CreationArmure} permet d'instancier une armure de base (décrite directement dans le code). Le plugin \textsc{CreationArmureFichier} prend un fichier texte (\texttt{.txt}) en paramêtre, et récupére les propriétés décrites dans ce fichier pour instancier une armure.


\section{Plugin de modification}
