%%%%%%%%%%%%%%%%%%%%%%%%%%%%%%%%%%%%%%%%%%%%%%%%%%%%%%%%%%%%%%%%%%%%
%%%%% DOC DEVELOPPEUR : FONCTIONNEMENT DES PLUGINS %%%%%
%%%%%%%%%%%%%%%%%%%%%%%%%%%%%%%%%%%%%%%%%%%%%%%%%%%%%%%%%%%%%%%%%%%%


L'\textsc{\'Editeur d'Armure} est un plugin qui se base sur un gestionnaire de plugins et sur plusieurs autres plugins pour fonctionner. \\


\section{GestionnairePlugins et ApplicationProjet}

Le gestionnaire de plugin sert à gérer tous les plugins (y compris l'ArmorEditor). Son comportement et son architecture sont détaillés dans le document concernant l'architecture. Le gestionnaire de plugin permet de lancer l'application, et aussi de charger des plugins. \\

L'ApplicationProjet, quant à lui, est le projet principal du logiciel. Il contient le plugin ArmorEditor, les classes qui permettent de décrire une armure, et les interfaces nécéssaires à l'ArmorEditor pour communiquer avec les plugins dont il a besoin (afficheur, modificateur \dots).


\section{Plugins d'affichage}

Les plugins d'affichage regroupent tous les plugins qui implémentent l'interface \texttt{IAfficheur}. Deux d'entre eux sont fournis actuellement.\\

Le plugin \texttt{AffichageConsole}, qui utilise simplement la methode \texttt{toString} sur l'instance de l'armure, et qui affiche les informations recueillies dans la console utilisée pour exécuter le logiciel. Le plugin \texttt{AffichageGraphique}, plus complexe, qui annalyse les données d'une armure pour les afficher proprement dans une fenêtre.


\section{Plugins de création}

Les plugins de création regroupent tous les plugins qui implémentent l'interface \texttt{ICreateur}. Deux d'entre eux sont fournis actuellement.\\

Le plugin \texttt{CreationArmure} permet d'instancier une armure de base (décrite directement dans le code). Le plugin \texttt{CreationArmureFichier} prend un fichier texte (\texttt{.txt}) en paramêtre, et récupére les propriétés décrites dans ce fichier pour instancier une armure. Le format du fichier texte est décrit plus précisément dans le tutoriel présent dans la documentation utilisateur.


\section{Plugin de modification}

Les plugins de modification regroupent tous les plugins qui implémentent l'interface \texttt{IModificateur}. Celui que nous avons créé contient un ensembles de méthodes capables de modifiées les propriétés d'une armure instanciée.
