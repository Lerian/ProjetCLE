%%%%%%%%%%%%%%%%%%%%%%%%%%%%%%%%%%%%%%%%%%%%%%%%%%%%%%%%%%%%%%%%%%%%
%%%%% DOC DEVELOPPEUR : TUTORIEL CREATION PLUGIN %%%%%
%%%%%%%%%%%%%%%%%%%%%%%%%%%%%%%%%%%%%%%%%%%%%%%%%%%%%%%%%%%%%%%%%%%%
Pour créer un nouveau plugin il est conseillé d'utilisé l'IDE Eclipse et d'y importer les projets disponible sur GitHub par la commande :
\vspace{0.5cm}\\
\begin{tabular}{|>{\columncolor{lightgray}}p{11.5cm}|}
	\hline
	\texttt{git clone https://github.com/Lerian/ProjetCLE.git}\\
	\hline
\end{tabular}\\

Ou en passant par le plugin git sous eclipse.
Une fois les projets correctements importé, vous pouvez commencé la création de votre plugin.

Pour commencé, créez un nouveau projet java à placer dans le dossier ProjetCLE.

Créez ensuite une classe qui sera votre classe principal.
Cette classe devra implémenter l'interface IPlugin du projet GestionnairePlugins.
Si vous souhaitez creez un plugin pour le logiciel d'édition d'armure, vous devez choisir le type de plugin à creez :
\begin{itemize}
    \item Créateur : implémenter l'interface ICreateur du projet ArmorEditor.
    \item Modificateur : implémenter l'interface IModificateur du projet ArmorEditor.
    \item Afficheur : implémenter l'interface IAfficheur du projet ArmorEditor.
\end{itemize}
\vspace{0.5cm}
Si des erreurs apparaisse ajouter les projets GestionnairePlugins et ArmorEditor à votre build path.

Vous pouvez maintenant créer votre plugin en implémentant les méthodes des interfaces.

La méthode run est executé lors de l'ajout de votre plugin au lancement de la plateforme.

La méthode type doit retourné le type du plugin.

La méthode receiveProperties fournie un objet Properties qui contient : "pathToHome", "pathToInit" ainsi que les informations qui seront à mettre dans un fichier .init à créer dans le dossier resources de votre projet.

Voila vous avez créer votre plugin, félicitation !
