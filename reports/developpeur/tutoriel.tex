%%%%%%%%%%%%%%%%%%%%%%%%%%%%%%%%%%%%%%%%%%%%%%%%%%%%%%%%%%%%%%%%%%%%
%%%%% DOC DEVELOPPEUR : TUTORIEL CREATION PLUGIN %%%%%
%%%%%%%%%%%%%%%%%%%%%%%%%%%%%%%%%%%%%%%%%%%%%%%%%%%%%%%%%%%%%%%%%%%%
Pour créer un nouveau plugin il est conseillé d'utilisé l'IDE Eclipse et d'y importer les projets disponible sur GitHub par la commande :
\vspace{0.5cm}\\
\begin{tabular}{|>{\columncolor{lightgray}}p{11.5cm}|}
	\hline
	\texttt{git clone https://github.com/Lerian/ProjetCLE.git}\\
	\hline
\end{tabular}\\

Ou en passant par le plugin git sous eclipse.
\vspace{0.5cm}

Une fois les projets correctements importé, vous pouvez commencé la création de votre plugin.
\vspace{0.5cm}

Pour commencé, créez un nouveau projet java.
\vspace{0.5cm}

Créez ensuite une classe qui sera votre classe principal.
Cette classe devra implémenter l'interface IPlugin du projet GestionnairePlugins.
\vspace{0.5cm}

Si vous souhaitez creez un plugin pour le logiciel d'édition d'armure, vous devez choisir le type de plugin à creez :
\begin{itemize}
    \item Créateur : implémenter l'interface ICreateur du projet ArmorEditor.
    \item Modificateur : implémenter l'interface IModificateur du projet ArmorEditor.
    \item Afficheur : implémenter l'interface IAfficheur du projet ArmorEditor.
\end{itemize}
\vspace{0.5cm}

Si des erreurs apparaisse ajouter les projets GestionnairePlugins et ArmorEditor à votre build path.
\vspace{0.5cm}

Vous pouvez maintenant créer votre plugin en implémentant les méthodes des interfaces.
\vspace{0.5cm}

La méthode run est executé lors de l'ajout de votre plugin au lancement de la plateforme.

La méthode type doit retourné le type du plugin qui est à définir, pour l'éditeur d'armure, il existe déjà 4 types défini dans l'enum PluginTypes : CREATEUR, AFFICHEUR, MODIFICATEUR et MAIN.

La méthode receiveProperties fournie un objet Properties qui contient : "pathToHome" qui retourne sous forme de string le chemin d'accès vers le dossier dans lequel est contenut le projet GestionnairePlugins, "pathToInit" qui retourne sous forme de string le chemin d'accès vers le fichier .init de votre projet ainsi que les informations qui peuvent vous être utils à mettre dans un fichier .init à créer dans le dossier resources de votre projet.
\vspace{0.5cm}

En ce qui concerne le programme d'édition d'armure, si vous souhaitez créer un afficheur, vous devez mettre le code principal de votre affichage dans la méthode affiche().

Pour le modificateur vous devez implementer les différentes méthodes qui permettents de changé les valeurs des armes et armures.

Et enfin pour le créateur, vous devez construire votre armure dans la méthode cree().
\vspace{0.5cm}

Il vous suffit ensuite de déclarer votre plugin, si vous construiser un plugin général, il est à déclaré dans le fichier init du gestionnaire de plugin pour le plugin principal (loadAtStart) ou à appellé par un autre plugin.

Par contre si vous souhaitez construire un nouveau composant pour l'éditeur d'armure, vous devez déclarer le plugin comme étant une dépendance de l'éditeur d'armure et donc le décalré dans armorEditor.init en suivant la syntaxe décrite.
\vspace{0.5cm}

Voila vous avez créer votre plugin, félicitation !
