%%%%%%%%%%%%%%%%%%%%%%%%%%%%%%%%%%%%%%%%%%%%%%%%%%%%%%%%%%%%%%%%%%%%
%%%%% DOC DEVELOPPEUR : INFORMATIONS GENERALES %%%%%
%%%%%%%%%%%%%%%%%%%%%%%%%%%%%%%%%%%%%%%%%%%%%%%%%%%%%%%%%%%%%%%%%%%%

L'\textsc{\'Editeur d'Armure} est un logiciel obéit à un scénario composé par les étudiants du Master 1 \textsc{Alma} : \\
un utilisateur quelconque souhaîte créer ou éditer des données, décrivant une armure de protection/combat (telle celle d'Iron Man).

Ce logiciel, à but pédagogique, a été conçu afin de permettre aux étudiants de parfaire leurs connaissances en \textit{Conception de logiciels extensibles} en créant leur propre architecture à plugins. 

\section{Récupération du code source}

Le code source de l’application est disponible sur le site GitHub. Pour le moment, le code source peut être visualisé et télécharger par tout le monde. Il est possible de le récupérer, sous forme \href{https://github.com/Lerian/ProjetCLE/archive/master.zip}{d'archive}, ou directement par l'intremédiaire de commandes git.
\vspace{0.5cm}\\
\begin{tabular}{|>{\columncolor{lightgray}}p{11.5cm}|}
	\hline
	\texttt{git clone https://github.com/Lerian/ProjetCLE.git}\\
	\hline
\end{tabular}\\

\section{Importation et architecture}

L'\textsc{\'Editeur d'Armure} est un logiciel conçus en \texttt{Java} 7 avec l'IDE Eclipse. Pour accéder au code source sans soucis, il est conseillé d'importer au préalable l'intégralité des projets sous Eclipse avant d'y faire des modifications.\\

\noindent{Comme il est dit précédement, le code source est divisé en plusieurs projets :}
\begin{itemize}
	\item Application projet : plugin principal
	\item GestionnairePlugins : gestionnaire de plugin indispensable au bon fonctionnement du logiciel.
	\item CreationArmure : plugin de création
	\item CreationArmureFichier : plugin de création
	\item ModificationArmure : plugin de modification de données
	\item AffichageConsole : plugin d'affichage
	\item AffichageGraphique : plugin d'affichage
\end{itemize}

