%%%%%%%%%%%%%%%%%%%%%%%%%%%%%%%%%%%%%%%%%%%%%%%%%%%%%%%%%%%%%%%%%%%%
%%%%% DOC DEVELOPPEUR : PREFACE %%%%%
%%%%%%%%%%%%%%%%%%%%%%%%%%%%%%%%%%%%%%%%%%%%%%%%%%%%%%%%%%%%%%%%%%%%


Les architectures à plugins sont aujourd'hui considérées comme étant l'avenir des logiciels. En effet, les plugins sont souvent la base d’une architecture logicielle modulaire, comme c’est le cas pour la plate-forme Eclipse et les bundles OSGi.\\

L'actuel Master \textsc{Alma} de l’Université de Nantes, offre la possibilité aux étudiants d'étudier la \textit{Conception de logiciels extensibles}. Les points abordés dans cette matière présentent plusieurs techniques d'architectures à plugins, permettant de développer des logiciels extensibles. \\

L'\textsc{\'Editeur d'Armure} se trouve être un logiciel extensible réalisé comme projet de fin de semestre du module \textit{Conception de logiciels extensibles}.\\

Cette documentation a été écrite dans le but de commenter le projet \textsc{\'Editeur d'Armure}. Chaque particularité du logiciel y est décrite ainsi que des exemples de manipulations afin de permettre à n’importe quel développeur de pouvoir créer un plugin compatible avec l'\textsc{\'Editeur d'Armure}.
