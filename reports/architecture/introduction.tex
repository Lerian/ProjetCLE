%%%%%%%%%%%%%%%%%%%%%%%%%%%%%%%%%%%%%%%%%%%%%%%%%%%%%%%%%%%%%%%%%%%%
%%%%% DOC ARCHITECTURE : INTRODUCTION %%%%%
%%%%%%%%%%%%%%%%%%%%%%%%%%%%%%%%%%%%%%%%%%%%%%%%%%%%%%%%%%%%%%%%%%%%


%%%%%
	%	Noter (en intro ?) la différence entre plateforme, gestionnaires de plugin ...
	%
	%
Une plateforme est une base de travail, à partir de laquelle il est possible d'utiliser un ensemble de logiciels. Un Gestionnaire de plugins, quant à lui, permet aux développeurs de gérer et d'installer de nouveaux plugins, en toute simplicité. Cela permet donc d'enrichir un logiciel, sans le surcharger, en lui rajoutant les fonctionnalité désirées. C'est aussi pourquoi les architectures à plugins sont aujourd'hui considérées comme étant l'avenir des logiciels. \\

L'actuel Master \textsc{Alma} de l’Université de Nantes, offre la possibilité aux étudiants d'étudier la \textit{Conception de logiciels extensibles}. Les points abordés dans cette matière présentent plusieurs techniques d'architectures à plugins, permettant de développer des logiciels extensibles. \\

L'\textsc{\'Editeur d'Armure} se trouve être un logiciel extensible réalisé comme projet de fin de semestre du module \textit{Conception de logiciels extensibles}.\\

Ce rapport a été écrit dans le but de commenter le projet \textsc{\'Editeur d'Armure}. Il présente donc les différentes étapes de notre travail, depuis l'imagination, jusqu'à la réalisation de notre projet.
