%%%%%%%%%%%%%%%%%%%%%%%%%%%%%%%%%%%%%%%%%%%%%%%%%%%%%%%%%%%%%%%%%%%%
%%%%% DOC UTILISATEUR : NOTIONS FONDAMENTALES %%%%%
%%%%%%%%%%%%%%%%%%%%%%%%%%%%%%%%%%%%%%%%%%%%%%%%%%%%%%%%%%%%%%%%%%%%


L'\textsc{\'Editeur d'Armure} est un logiciel créé par des étudiants du Master 1 \textsc{Alma}. il permet de créer des armures composées de différentes pièces, puis de les afficher et de les modifier. Bien que chacun puisse créer de nouveaux plugins dans le but de donner à l'éditeur d'armure le comportement souhaité, ce document aura pour but de présenter les plugins mis à disposition avec le logiciel. Couvrant toutes les fonctionnalités possibles (création, affichage et modification d'armure), ces plugins permettent à chacun d'utiliser l'éditeur d'armure sans avoir besoin de créer de code.

\section{Récupération de l'application}

Le code source de l’application est disponible sur un dépôt Git hébergé par le site GitHub (\href{https://github.com/lerian/ProjetCLE}{github.com/lerian/ProjetCLE}). Le code source peut être visualisé et téléchargé par tout le monde. Il est possible de le récupérer, sous forme \href{https://github.com/Lerian/ProjetCLE/archive/master.zip}{d'archive} ou directement par l'intermédiaire de Git:\\

\begin{tabular}{|>{\columncolor{lightgray}}p{11.5cm}|}
	\hline
	\texttt{git clone https://github.com/Lerian/ProjetCLE.git}\\
	\hline
\end{tabular}\\

\section{Importation et exécution}

L'\textsc{\'Editeur d'Armure} est un logiciel conçus en \texttt{Java} 7. Ayant été développé avec l'IDE Eclipse, la façon la plus simple de l'exécuter reste d'importer les projets avec Eclipse.\\

Une fois tous les projets importés, il faut indiquer à l'éditeur d'armure les tâches qu'il doit effectuer. Ceci peut se faire de deux façons différentes, qui sont les suivantes:\\

\begin{itemize}
	\item En modifiant le code de la méthode \texttt{public void run()} de la classe \texttt{ArmorEditor}.
	\item En modifiant le fichier de configuration de l'éditeur d'armure. 
\end{itemize}
