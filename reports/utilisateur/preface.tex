%%%%%%%%%%%%%%%%%%%%%%%%%%%%%%%%%%%%%%%%%%%%%%%%%%%%%%%%%%%%%%%%%%%%
%%%%% DOC UTILISATEUR : PREFACE %%%%%
%%%%%%%%%%%%%%%%%%%%%%%%%%%%%%%%%%%%%%%%%%%%%%%%%%%%%%%%%%%%%%%%%%%%


Les architectures à plugins sont aujourd'hui considérées comme étant l'avenir des logiciels. En effet, un plugin est un paquet qui complète un logiciel hôte pour lui apporter de nouvelles fonctionnalités. Ainsi, ils sont souvent la base d’une architecture logicielle modulaire, comme c’est le cas pour la plate-forme Eclipse.\\

L'actuel Master \textsc{Alma} de l’Université de Nantes, offre la possibilité aux étudiants d'étudier la \textit{Conception de logiciels extensibles}. Les points abordés dans cette matière présentent plusieurs techniques d'architectures à plugins, permettant de développer des logiciels extensibles. \\

L'\textsc{\'Editeur d'Armure} se trouve être un logiciel extensible réalisé comme projet de fin de semestre du module \textit{Conception de logiciels extensibles}.\\

Cette documentation a été écrite dans le but de commenter le projet \textsc{\'Editeur d'Armure}. Chaque caractéristique du logiciel y est décrit afin de permettre à n’importe quel utilisateur de pouvoir se servir du logiciel \textsc{\'Editeur d'Armure}.
