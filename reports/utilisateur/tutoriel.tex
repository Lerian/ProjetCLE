%%%%%%%%%%%%%%%%%%%%%%%%%%%%%%%%%%%%%%%%%%%%%%%%%%%%%%%%%%%%%%%%%%%%
%%%%% DOC UTILISATEUR : TUTORIEL %%%%%
%%%%%%%%%%%%%%%%%%%%%%%%%%%%%%%%%%%%%%%%%%%%%%%%%%%%%%%%%%%%%%%%%%%%

L'\textsc{\'Editeur d'Armure} offre la possibilité de créer une armure à partir d'un fichier texte. Pour cela, il faut au préalable : 
\begin{itemize}
	\item créer un fichier texte de description, compatible avec le plugin "CreationArmureFichier".
	\item configurer le logiciel pour qu'il utilise le bon plugin, et le bon fichier.
\end{itemize}


\subsection{Création du fichier texte}

Avant tout, il faut créer un fichier qui servira de base pour le plugin de création d'armure. Ce fichier doit être au format \texttt{.txt}, nous prendrons pour exemple \texttt{IronMan.txt}.\\

Ensuite, il faut y décrire l'armure qui doit être créée. Le plugin "CreationArmureFichier" déchiffre le fichier texte grâce à des mots clefs (properties) tels que \texttt{nomArmure} ou \texttt{valEnergie}. Une armure est composée de divers éléments :
\begin{itemize}
	\item un nom;
	\item une énergie;
	\item 6 morceaux d'armure (casque, torse, jambes et bras gauche et droit);
	\item 2 armes;
	\item une image (facultatif).
\end{itemize}

\subsection{Configuration de l'application}

[[[[[[[[[[[[[[[[[[[[[[[ A COMPLETER ]]]]]]]]]]]]]]]]]]]]]]]
